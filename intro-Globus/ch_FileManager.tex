\chapter{Managing and transferring files}

\section{The File Manager}\label{file-manager}

After you have signed up and logged in to Globus, you will begin at the \emph{File Manager}. The first time you use the \emph{File Manager}, all fields will be blank.

\begin{center}
\includegraphics[width=0.5\textwidth]{img/filemanager-1.png}
\end{center}

\section{Collections and Endpoints}\label{collections-endpoints}

A \gls{Collection} is a named location containing data you can access with Globus. Collections can be hosted on many different kinds of systems, including campus storage, HPC clusters, laptops, Amazon S3 buckets, Google Drive, and scientific instruments. When you use Globus, you do not need to know a physical location or details about storage. You only need a collection name. A collection allows authorized Globus users to browse and transfer files. Collections can also be used for sharing data with others, for data publication, and for enabling discovery by other Globus users. \gls{Globus Connect} is used to host collections.

An \gls{Endpoint} is a server that hosts collections. If you want to be able to access, share, transfer, or manage data using Globus, the first step is to create an endpoint on the system where the data is (or will be) stored.

To access a Collection
\begin{itemize}
\item Click in the \gls{Collection} field at the top of the File Manager page and search available collections and endpoints by typing a collection/endpoint name or a description. Globus will list collections with matching names.
\item Click on a collection. Globus will connect to the collection and display the default directory. Click the \emph{Path} field and change the path if needed. Globus will show the files in the new path.
\end{itemize}

\section{Transferring files}\label{file-transfer}

\begin{center}
\includegraphics[width=0.5\textwidth]{img/filetransfer-2.png}
\end{center}

\begin{itemize}
\item Click \emph{Transfer} or \emph{Sync to...} in the command panel on the right side of the page. A new collection panel will open, with a \emph{Transfer or Sync to} field at the top of the panel.
\item Find and choose the second collection and connect to it as you did with the Globus with the first one. Click on the left first collection and select all the files to transfer. The \emph{Start} button at the bottom of the panel will activate.
Between the two \emph{Start} buttons at the bottom of the page, the \emph{Transfer and Sync Options} tab provides access to several options. By default, Globus verifies file integrity after transfer using checksums. Change the transfer settings if you would like. You may also enter a label for the transfer.
\item Click the \emph{Start} button to transfer the selected files to the collection in the right panel. Globus will display a green notification panel - confirming that the transfer request was submitted - and add a badge to the Activity item in the command menu on the left of the page.
You can navigate away from the \emph{File Manager}, close the browser window, and even logout. Globus will optimize the transfer for performance, monitor the transfer for completion and correctness, and recover from network errors and collection downtime.
\end{itemize}

Completed file transfers can be seen in the \emph{Activity} tab in the command menu on the left of the page. On the Activity page, click the arrow icon on the right to view details about the transfer. You will also receive an email with the transfer details.

%%% Local Variables:
%%% mode: latex
%%% TeX-master: "intro-Globus"
%%% End:
