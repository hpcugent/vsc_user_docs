\chapter{Multi-job submission}
\label{ch:multi-job-submission}

A frequent occurring characteristic of scientific computation is their focus on
data intensive processing. A typical example is the iterative evaluation of a
program over different input parameter values, often referred to as a
``\emph{parameter sweep}''.  A \strong{Parameter Sweep} runs a job a
specified number of times, as if we sweep the parameter values through a user
defined range.

Users then often want to submit a large numbers of jobs based on the same job
script but with (i) slightly different parameters settings or with (ii)
different input files.

These parameter values can have many forms, we can think about a range (e.g.,
from 1 to 100), or the parameters can be stored line by line in a
comma-separated file. The users want to run their job once for each instance of
the parameter values.

One option could be to launch a lot of separate individual small jobs (one for
each parameter) on the cluster, but this is not a good idea. The cluster
scheduler isn't meant to deal with tons of small jobs. Those huge amounts of
small jobs will create a lot of overhead, and can slow down the whole cluster.
It would be better to bundle those jobs in larger sets.  In TORQUE, an
experimental feature known as ``\emph{job arrays}'' existed to allow the
creation of multiple jobs with one \emph{qsub} command, but is was not
supported by Moab, the current scheduler.

The ``\strong{Worker framework}'' has been developed to address this issue.

It can handle many small jobs determined by:

\begin{description}
  \item[parameter variations] i.e., many small jobs determined by a specific parameter set which is stored in a .csv (comma separated value) input file.
  \item[job arrays] i.e., each individual job got a unique numeric identifier.
\end{description}

Both use cases often have a common root: the user wants to run a program with a
large number of parameter settings, and the program does not allow for
aggregation, i.e., it has to be run once for each instance of the parameter
values.

However, the Worker Framework's scope is wider: it can be used for any scenario
that can be reduced to a \strong{MapReduce} approach.\footnote{MapReduce:
`Map' refers to the map pattern in which every item in a collection is mapped
onto a new value by applying a given function, while ``reduce'' refers to the
reduction pattern which condenses or reduces a collection of previously
computed results to a single value.}

\section{The worker Framework: Parameter Sweeps}

First go to the right directory:

\begin{prompt}
%\shellcmd{cd ~/\exampledir/par\_sweep}%
\end{prompt}

Suppose the program the user wishes to run the ``\emph{weather}'' program,
which takes three parameters: a temperature, a pressure and a volume. A typical
call of the program looks like:

\begin{prompt}
%\shellcmd{./weather -t 20 -p 1.05 -v 4.3}%
T: 20  P: 1.05  V: 4.3
\end{prompt}

For the purpose of this exercise, the weather program is just a simple bash
script, which prints the 3 variables to the standard output and waits a bit:

\examplecode{bash}{par_sweep/weather}

A job-script that would run this as a job for the first parameters (p01) would
then look like:

\examplecode{bash}{par_sweep/weather_p01.pbs}

When submitting this job, the calculation is performed or this particular
instance of the parameters, i.e., temperature = 20, pressure = 1.05, and volume
= 4.3.

To submit the job, the user would use:

\begin{prompt}
%\shellcmd{qsub weather\_p01.pbs}%
\end{prompt}

However, the user wants to run this program for many parameter instances, e.g.,
he wants to run the program on 100 instances of temperature, pressure and
volume.  The 100 parameter instances can be stored in a comma separated value
file (.csv) that can be generated using a spreadsheet program such as Microsoft
Excel or RDBMS or just by hand using any text editor (do \strong{not} use a
word processor such as Microsoft Word). The first few lines of the file
``\emph{data.csv}'' would look like:

\begin{prompt}
%\shellcmd{more data.csv}%
temperature, pressure, volume
293, 1.0e5, 107
294, 1.0e5, 106
295, 1.0e5, 105
296, 1.0e5, 104
297, 1.0e5, 103
%\dots%
\end{prompt}

It has to contain the names of the variables on the first line, followed by 100
parameter instances in the current example.

In order to make our PBS generic, the PBS file can be modified as follows:

\examplecode{bash}{par_sweep/weather.pbs}

Note that:

\begin{enumerate}
  \item  the parameter values 20, 1.05, 4.3 have been replaced by variables
    \$temperature, \$pressure and \$volume respectively, which were being specified
    on the first line of the ``\emph{data.csv}'' file;
  \item  the number of processors per node has been increased to 8 (i.e., ppn=1 is replaced by
    ppn=8);
  \item  the walltime has been increased to 4 hours (i.e.,
    walltime=00:15:00 is replaced by walltime=04:00:00).
\end{enumerate}

The walltime is calculated as follows: one calculation takes 15 minutes, so 100
calculations take 1500 minutes on one CPU. However, this job will use 8 CPUs,
so the 100 calculations will be done in 1500/8 = 187.5 minutes, i.e., 4 hours
to be on the safe side.

The job can now be submitted as follows:

\begin{prompt}
%\shellcmd{module load worker}%
%\shellcmd{wsub -batch weather.pbs -data data.csv}%
total number of work items: 41
%\jobid%
\end{prompt}

Note that the PBS file is the value of the -batch option. The weather program
will now be run for all 100 parameter instances -- 8 concurrently -- until
all computations are done. A computation for such a parameter instance is
called a work item in Worker parlance.

\section{The Worker framework: Job arrays}

First go to the right directory:

\begin{prompt}
%\shellcmd{cd ~/\exampledir/job\_array}%
\end{prompt}

As a simple example, assume you have a serial program called \emph{myprog} that you
want to run on various input files \textit{input[1-100]}.

\begin{center}
\includegraphics*[width=3.22in, height=2.36in, keepaspectratio=false]{img0702}
\end{center}

The following bash script would submit these jobs all one by one:
\begin{code}{bash}
#!/bin/bash
for i in `seq 1 100`; do
   qsub -o output $i -i input $i myprog.pbs
done
\end{code}

This, as said before, could be disturbing for the job scheduler.

Alternatively, TORQUE provides a feature known as \emph{job arrays} which
allows the creation of multiple, similar jobs with only \strong{one qsub}
command. This feature introduced a new job naming convention that allows users
either to reference the entire set of jobs as a unit or to reference one
particular job from the set.

Under TORQUE, the \emph{-t range} option is used with qsub to specify a job
array, where \emph{range} is a range of numbers (e.g., \emph{1-100} or
\emph{2,4-5,7}).

The details are

\begin{enumerate}
\item  a job is submitted for each \emph{number} in the range;
\item  individuals jobs are referenced as \emph{jobid-number}, and the entire array can be referenced as \emph{jobid} for easy killing etc.; and
\item  each jobs has \emph{PBS\_ARRAYID} set to its \emph{number} which allows the script/program to specialise for that job
\end{enumerate}

The job could have been submitted using:

\begin{prompt}
%\shellcmd{qsub -t 1-100  my\_prog.pbs}%
\end{prompt}

The effect was that rather than 1 job, the user would actually submit 100 jobs
to the queue system. This was a popular feature of TORQUE, but as this
technique puts quite a burden on the scheduler, it is not supported by Moab
(the current job scheduler).

To support those users who used the feature and since it offers a convenient
workflow, the ``worker framework'' implements the idea of ``job arrays'' in its
own way.

A typical job-script for use with job arrays would look like this:

\examplecode{bash}{job_array/job_array.pbs}

In our specific example, we have prefabricated 100 input files in the
``./input'' subdirectory. Each of those files contains a number of parameters
for the ``test\_set'' program, which will perform some tests with those
parameters.

Input for the program is stored in files with names such as input\_1.dat,
input\_2.dat, \ldots, input\_100.dat in the ./input subdirectory.

\begin{prompt}
%\shellcmd{ls ./input}%
%\dots%
%\shellcmd{more ./input/input\_99.dat}%
This is input file \#99
Parameter #1 = 99
Parameter #2 = 25.67
Parameter #3 = Batch
Parameter #4 = 0x562867
\end{prompt}

For the sole purpose of this exercise, we have provided a short ``test\_set''
program, which reads the ``input'' files and just copies them into a
corresponding output file.  We even add a few lines to each output file. The
corresponding output computed by our ``\textit{test\_set}'' program will be
written to the \textit{``./output}'' directory in output\_1.dat, output\_2.dat,
\ldots, output\_100.dat. files.

\examplecode{bash}{job_array/test_set}

Using the ``worker framework'', a feature akin to job arrays can be used with
minimal modifications to the job-script:

\examplecode{bash}{job_array/test_set.pbs}

Note that

\begin{enumerate}
\item  the number of CPUs is increased to 8 (ppn=1 is replaced by ppn=8); and
\item  the walltime has been modified (walltime=00:15:00 is replaced by walltime=04:00:00).
\end{enumerate}

The job is now submitted as follows:

\begin{prompt}
%\shellcmd{module load worker}%
%\shellcmd{wsub -t 1-100 -batch test\_set.pbs}%
total number of work items: 100
%\jobid%
\end{prompt}

The ``\textit{test\_set}'' program will now be run for all 100 input files --
8 concurrently -- until all computations are done. Again, a computation for an
individual input file, or, equivalently, an array id, is called a work item in
Worker speak.

Note that in contrast to TORQUE job arrays, a worker job array only submits a
single job.

\begin{prompt}
%\shellcmd{qstat}%
Job id          Name          User      Time   Use S Queue
--------------- ------------- --------- ---- ----- - -----
%\jobid%  test_set.pbs  %\userid%          0 Q

And you can now check the generated output files:
%\shellcmd{more ./output/output\_99.dat}%
This is output file #99
Calculations done, no results
\end{prompt}

\section{MapReduce: prologues and epilogue}

Often, an embarrassingly parallel computation can be abstracted to three simple
steps:

\begin{enumerate}
\item  a preparation phase in which the data is split up into smaller, more manageable chunks;
\item  on these chunks, the same algorithm is applied independently (these are the work items); and
\item  the results of the computations on those chunks are aggregated into, e.g., a statistical description of some sort.
\end{enumerate}

\begin{center}
\includegraphics*[width=2.40in, height=2.78in, keepaspectratio=false]{img0703}
\end{center}

The Worker framework directly supports this scenario by using a prologue
(pre-processing) and an epilogue (post-processing). The former is executed just
once before work is started on the work items, the latter is executed just once
after the work on all work items has finished. Technically, the master, i.e.,
the process that is responsible for dispatching work and logging progress,
executes the prologue and epilogue.

\begin{prompt}
%\shellcmd{cd ~/\exampledir/map\_reduce}%
\end{prompt}

The script ``pre.sh'' prepares the data by creating 100 different input-files,
and the script ``post.sh'' aggregates (concatenates) the data.


First study the scripts:

\examplecode{bash}{map_reduce/pre.sh}
\examplecode{bash}{map_reduce/post.sh}

Then one can submit a MapReduce style job as follows:

\begin{prompt}
%\shellcmd{wsub}% -prolog pre.sh -batch test_set.pbs -epilog post.sh -t 1-100
total number of work items: 100
%\jobid%
%\shellcmd{cat all\_output.txt}%
%\dots%
%\shellcmd{rm -r -f ./output/}%
\end{prompt}

Note that the time taken for executing the prologue and the epilogue should be
added to the job's total walltime.

\section{Some more on the Worker Framework}

\subsection{Using Worker efficiently}

The ``Worker Framework'' is implemented using MPI, so it is not restricted to a
single compute nodes, it scales well to multiple nodes. However, remember that
jobs requesting a large number of nodes typically spend quite some time in the
queue.

The ``Worker Framework'' will be effective when

\begin{enumerate}
\item  work items, i.e., individual computations, are neither too short, nor too long (i.e., from a few minutes to a few hours); and,
\item  when the number of work items is larger than the number of CPUs involved in the job (e.g., more than 30 for 8 CPUs).
\end{enumerate}

\subsection{Monitoring a worker job}

Since a Worker job will typically run for several hours, it may be reassuring
to monitor its progress. Worker keeps a log of its activity in the directory
where the job was submitted. The log's name is derived from the job's name and
the job's ID, i.e., it has the form $<$jobname$>$.log$<$jobid$>$. For the
running example, this could be ``run.pbs.log\jobnumber'', assuming the job's ID is
\jobnumber. To keep an eye on the progress, one can use:

\begin{prompt}
%\shellcmd{tail -f run.pbs.log\jobnumber}%
\end{prompt}

Alternatively, ``wsummarize'', a Worker command that summarises a log file, can be used:

\begin{prompt}
%\shellcmd{watch -n 60 wsummarize run.pbs.log\jobnumber}%
\end{prompt}

This will summarise the log file every 60 seconds.

\subsection{Time limits for work items}

Sometimes, the execution of a work item takes long than expected, or worse,
some work items get stuck in an infinite loop. This situation is unfortunate,
since it implies that work items that could successfully execute are not even
started. Again, the Worker framework offers a simple and yet versatile
solution. If we want to limit the execution of each work item to at most 20
minutes, this can be accomplished by modifying the script of the running
example.

\begin{code}{bash}
#!/bin/bash -l
#PBS -l nodes=1:ppn=8
#PBS -l walltime=04:00:00
module load timedrun/1.0
cd $PBS_O_WORKDIR
timedrun -t 00:20:00 weather -t $temperature  -p $pressure  -v $volume
\end{code}

Note that it is trivial to set individual time constraints for work items by
introducing a parameter, and including the values of the latter in the CSV
file, along with those for the temperature, pressure and volume.

Also note that ``timedrun'' is in fact offered in a module of its own, so it can
be used outside the Worker framework as well.

\subsection{Resuming a Worker job}

Unfortunately, it is not always easy to estimate the walltime for a job, and
consequently, sometimes the latter is underestimated. When using the Worker
framework, this implies that not all work items will have been processed.
Worker makes it very easy to resume such a job without having to figure out
which work items did complete successfully, and which remain to be computed.
Suppose the job that did not complete all its work items had ID ``445948''.

\begin{prompt}
%\shellcmd{wresume -jobid \jobnumber}%
\end{prompt}

This will submit a new job that will start to work on the work items that were
not done yet. Note that it is possible to change almost all job parameters when
resuming, specifically the requested resources such as the number of cores and
the walltime.

\begin{prompt}
%\shellcmd{wresume -l walltime=1:30:00 -jobid \jobnumber}%
\end{prompt}

Work items may fail to complete successfully for a variety of reasons, e.g., a
data file that is missing, a (minor) programming error, etc. Upon resuming a
job, the work items that failed are considered to be done, so resuming a job
will only execute work items that did not terminate either successfully, or
reporting a failure. It is also possible to retry work items that failed
(preferably after the glitch why they failed was fixed).

\begin{prompt}
%\shellcmd{wresume -jobid \jobnumber -retry}%
\end{prompt}

By default, a job's prologue is not executed when it is resumed, while its
epilogue is. ``wresume'' has options to modify this default behaviour.

\subsection{Further information}

This how-to introduces only Worker's basic features. The wsub command has some
usage information that is printed when the -help option is specified:

\begin{prompt}
%\shellcmd{wsub -help}%
### usage: wsub  -batch <batch-file>          \
#                [-data <data-files>]         \
#                [-prolog <prolog-file>]      \
#                [-epilog <epilog-file>]      \
#                [-log <log-file>]            \
#                [-mpiverbose]                \
#                [-dryrun] [-verbose]         \
#                [-quiet] [-help]             \
#                [-t <array-req>]             \
#                [<pbs-qsub-options>]
#
#   -batch <batch-file>   : batch file template, containing variables to be
#                           replaced with data from the data file(s) or the
#                           PBS array request option
#   -data <data-files>    : comma-separated list of data files (default CSV
#                           files) used to provide the data for the work
#                           items
#   -prolog <prolog-file> : prolog script to be executed before any of the
#                           work items are executed
#   -epilog <epilog-file> : epilog script to be executed after all the work
#                           items are executed
#   -mpiverbose           : pass verbose flag to the underlying MPI program
#   -verbose              : feedback information is written to standard error
#   -dryrun               : run without actually submitting the job, useful
#   -quiet                : don't show information
#   -help                 : print this help message
#   -t <array-req>        : qsub's PBS array request options, e.g., 1-10
#   <pbs-qsub-options>    : options passed on to the queue submission
#                           command
\end{prompt}
