\chapter{Graphical applications with VNC}
\label{ch:vnc}

It's important to remember that VNC sessions are permanent. They survive network
problems and (unintended) connection loss. This means you can logout and go home
without a problem (like the terminal equivalent \lstinline|screen| or \lstinline|tmux|).
This also means you don't have to start \lstinline|vncserver| each time you want to connect.

\section{Start the VNC server}
\label{sec:start-vnc}

First login on the login node (see \autoref{sec:first-time-connection-to-the-hpc})
or to the worker node where your job has started. % TODO: where is this described?

Now start \lstinline|vcnserver| with:

\begin{prompt}
%\shellcmd{vncserver -geometry 1920x1080}%
You will require a password to access your desktops.

Password:%\emph{<{}enter a secure password>{}}%
Verify:%\emph{<{}enter the same password>{}}%
Would you like to enter a view-only password (y/n)? %\emph{n}%
A view-only password is not used

New '%\strong{gligar02}%.gligar.os:%\strong{6}% (%\userid{}%)' desktop is gligar02.gligar.os:6

Creating default startup script /user/home/gent/vsc000/%\userid{}%/.vnc/xstartup
Creating default config /user/home/gent/vsc000/%\userid{}%/.vnc/config
Starting applications specified in /user/home/gent/vsc000/%\userid{}%/.vnc/xstartup
Log file is /user/home/gent/vsc000/%\userid{}%/.vnc/gligar02.gligar.os:6.log

\end{prompt}

When prompted for a password, make sure to enter a secure password: if someone
can guess your password, they will be able to do anything with your account you can.

Note down the details in bold: the hostname (in the example \lstinline|gligar02|)
and the port number (in the example \lstinline|6|).

\section{Connecting to the VNC server}

\subsection{VNC on the login nodes}

The VNC server runs on a \strong{specific login node} (in this example \lstinline|gligar02|).
Make sure you connect to this login node: the domain should be like \lstinline|gligar02.ugent.be|,
but the number can be different. If you're not sure how to do this, please follow the steps
in \autoref{sec:first-time-connection-to-the-hpc}, but replace \loginnode with your specific
domain (here \lstinline|gligar02.ugent.be|).

Login nodes are rebooted from time to time. You can check that the VNC server is still
running in the same note by executing `vncserver -list`. If you get an empty list,
it means that \lstinline|vncserver| is not running. You will need to start it again,
see \autoref{sec:start-vnc}.

You will now need to portforward the VNC port. The source port is the sum of \lstinline|5900|
and the number we noted down earlier. In this case, it would be \lstinline|5906|.
The destination port is the same as the source port. The host is \lstinline|localhost|.

\ifwindows
See \autoref{par:ssh-tunnel-windows}. Use the details specified here (host, destination port,
source port).
\else

Execute the following command. Make sure to replace the port numbers, userid and login node
with your own.

\begin{prompt}
%shellcmd{ssh -L 5906:localhost:5906 \userid{}@gligar02.ugent.be}
\end{prompt}
\fi

\ifwindows

You can download a free VNC client from \url{https://sourceforge.net/projects/turbovnc/files/}.
You can download the latest version by clicking the top-most folder that has a version number
in it that doesn't also have \lstinline|beta| in the version. Then download a file that looks like
\lstinline|TurboVNC64-2.1.2.exe| (the version number can be different, but the \lstinline|64|
should be in the filename) and execute it.
\fi
\ifmac
You can download a free VNC client from \url{https://sourceforge.net/projects/turbovnc/files/}.
You can download the latest version by clicking the top-most folder that has a version number
in it that doesn't also have \lstinline|beta| in the version. Then download a file ending in
\lstinline|TurboVNC64-2.1.2.dmg| (the version number can be different) and execute it.
\fi
\iflinux
Download and setup a VNC client. A good choice is \lstinline|tigervnc|. You can start
it with the \lstinline|vncviewer| command.
\fi

Now start your VNC client and connect to \lstinline|localhost:5906|, again replacing
the port with your own.

When promted for a password, use the password you used to setup the VNC server.
When prompted for default or empty panel, choose default.

If you have an empty panel, you can reset your settings with the following commands:

\begin{prompt}
%\shellcmd{xfce4-panel --quit ; pkill xfconfd}%
%\shellcmd{mkdir ~/.oldxfcesettings}%
%\shellcmd{mv ~/.config/xfce4 ~/.oldxfcesettings}%
%\shellcmd{xfce4-panel}%
\end{prompt}

\subsection{VNC on the worker nodes}

Connecting to a VNC server on the worker nodes is exactly the same as connecting
to a login node, except that you need to use the hostname of the worker node instead of
\lstinline|localhost| when setting up the SSH tunnel. A hostname of a worker node looks like
\texttt{\computenode{}}.

Note that you still need to use \lstinline|localhost| when connecting to the VNC server
with your VNC client.

\section{Stopping the VNC server}

The VNC server can be killed by running

\begin{prompt}
vncserver -kill :6
\end{prompt}

where \lstinline|6| is the port number we noted down earlier. If you forgot,
you can get it with \lstinline|vncserver -list|.

\section{I forgot the password, what now?}

You can reset the password by first stopping the VNC server, then removing
the \lstinline|.vnc/passwd| file (with \lstinline|rm .vnc/passwd|) and then
starting the VNC server again.
