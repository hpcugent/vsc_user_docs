\chapter{Best Practices}
\label{ch:best-practices}

\section{General Best Practices}
\label{sec:general-best-practices}
\begin{enumerate}

  \item  Before starting you should always check:
  \begin{enumerate}
    \item  Are there any errors in the script?
    \item  Are the required modules loaded?
    \item  Is the correct executable used?
  \end{enumerate}

  \item  Check your computer requirements upfront, and request the correct resources in your PBS configuration script.
  \begin{enumerate}
    \item  Number of requested cores
    \item  Amount of requested memory
    \item  Requested network type
  \end{enumerate}

  \item  Check your jobs at runtime. You could login to the node and check the
    proper execution of your jobs with, e.g., ``top'' or ''vmstat''.
    Alternatively you could run an interactive job (``qsub -I'').

  \item  Try to benchmark the software for scaling issues when using MPI or for
    I/O issues.

  \item  Use the scratch file system (\$VSC\_SCRATCH\_NODE which is mapped to the
    local /tmp) whenever possible. Local disk I/O is always much faster as it
    does not have to use the network.

  \item  When your job starts, it will log on to the compute node(s) and start
    executing the commands in the job script. It will start in your home
    directory (\$VSC\_HOME), so going to the current directory with the ``cd
    \$PBS\_O\_WORKDIR'' is the first thing which needs to be done.  You will
    have your default environment, so don't forget to load the software with
    ``module load''.

  \item  In case your job not running, use ``checkjob''.  It will show why your
    job is not yet running. Sometimes commands might timeout with an overloaded
    scheduler.

  \item  Submit your job and wait (be patient) \ldots

  \item  Submit small jobs by grouping them together. The ``Worker Framework''
    has been designed for these purposes.

  \item  The runtime is limited by the maximum walltime of the queues. For
    longer walltimes, use checkpointing.

  \item  Requesting many processors could imply long queue times.

  \item  For all parallel computing, request to use ``Infiniband''.

  \item  And above all \dots\ do not hesitate to contact the \hpc staff. We're
    here to help you.
\end{enumerate}

\section{Windows / Unix}

Important note: the PBS file on the \hpc has to be in UNIX format, if it is
not, your job will fail and generate rather weird error messages.

If necessary, you can convert it using
\begin{prompt}
%\shellcmd{dos2unix file.pbs}%
\end{prompt}

% ==================================================================================================

\section{Best Practices for EasyBuild}
\label{sec:best-practices-easybuild}

\textit{(coming soon)}

% ==================================================================================================

\section{Best Practices for \texttt{mympirun}}
\label{sec:best-practices-mympirun}
\label{sec:best-practices-mympirun-module}

\textit{(coming soon)}

\textit{for now, see {\small\url{https://github.com/hpcugent/vsc-mympirun/blob/master/README.md}}}

% ==================================================================================================

\section{Best practices for OpenFOAM}
\label{sec:best-practices-openfoam}

In this section, we outline best practices for using the centrally provided OpenFOAM installations
on the VSC \hpc infrastructure.

\textit{last update}: September 2017

\textit{authors}: Kenneth Hoste (HPC-UGent), with feedback from Joris Degroote (UGent), Brecht Devolder (UGent),
                  Pieter Reyniers (UGent), Laurien Vandewalle (UGent)


\subsection{Different OpenFOAM releases}
\label{sec:best-practices-openfoam-releases}

There are currently three different sets of versions of OpenFOAM available, each with its own versioning scheme:

\begin{itemize}
    \item OpenFOAM versions released via \url{http://openfoam.com}: \texttt{v3.0+}, \texttt{v1706}
    \begin{itemize}
        \item see also \url{http://openfoam.com/history/}
    \end{itemize}
    \item OpenFOAM versions released via \url{https://openfoam.org}: \texttt{v4.1}, \texttt{v5.0}
    \begin{itemize}
        \item see also \url{https://openfoam.org/download/history/}
    \end{itemize}
    \item OpenFOAM versions released via \url{http://wikki.gridcore.se/foam-extend}: \texttt{v3.1}
\end{itemize}

Make sure you known which flavor of OpenFOAM you want to use, since there are important differences between
the different versions w.r.t.\ features.

\subsection{Documentation \& training material}
\label{sec:best-practices-openfoam-documentation}

The best practices outlined here focus specifically on the use of OpenFOAM on the VSC \hpc infrastructure.
As such, they are intented to augment the existing OpenFOAM documentation rather than replace it.

For more general information on using OpenFOAM, please refer to:

\begin{itemize}
\item OpenFOAM websites:
\begin{itemize}
    \item \url{http://openfoam.com}
    \item \url{https://openfoam.org}
    \item \url{http://wikki.gridcore.se/foam-extend}
\end{itemize}
\item OpenFOAM user guides:
    \begin{itemize}
    \item \url{http://www.openfoam.com/documentation/user-guide}
    \item \url{https://cfd.direct/openfoam/user-guide/}
    \end{itemize}
\item recordings of "\textit{Introduction to OpenFOAM}" training session at UGent (May 2016):\\
      \small{\url{https://www.youtube.com/playlist?list=PLqxhJj6bcnY9RoIgzeF6xDh5L9bbeK3BL}}
\end{itemize}


\subsection{Preparing the environment}
\label{sec:best-practices-openfoam-environment}

To prepare your the environment of your shell session or job for OpenFOAM, there are a couple of things to take into
account.


\subsubsection{Picking and loading an \texttt{OpenFOAM} module}

First of all, you must pick and load one of the available \texttt{OpenFOAM} modules.

To get an overview of the available modules, run `{\small\texttt{module avail OpenFOAM}}'. For example:

\begin{prompt}
%\shellcmd{module avail OpenFOAM}%
% %
%----- /apps/gent/CO7/sandybridge/modules/all -----%
%   OpenFOAM/2.4.0-intel-2017a%
%   OpenFOAM/3.0.1-intel-2016b%
%   OpenFOAM/4.0-intel-2016b%
\end{prompt}

To pick a module, take into account the differences between the different OpenFOAM versions w.r.t.\ features and
API (see also Section~\ref{sec:best-practices-openfoam-releases}).

If multiple modules are available that fulfill your requirements, give preference to those providing a more recent
OpenFOAM version, and to the ones that were installed with a more recent compiler toolchain, i.e., prefer
a module that includes `{\small\texttt{intel-2017a}}' in its name over ones that includes
`{\small\texttt{intel-2016b}}'.

To prepare your environment for using OpenFOAM, load the \texttt{OpenFOAM} module you have picked:

\begin{prompt}
%module load OpenFOAM/4.0-intel-2016b%
\end{prompt}

\subsubsection{Sourcing the \texttt{\$FOAM\_BASH} script}

OpenFOAM provides a script that you should \texttt{\small{source}} to further prepare the environment.
This script will define some additional environment variables that are required to use OpenFOAM. The \texttt{OpenFOAM}
modules define an environment variable named `\texttt{\small\$FOAM\_BASH}' that specifies the location to this script.

Assuming you are using \texttt{\small{bash}} in your shell session or job script,
you must always run the following command after loading an \texttt{OpenFOAM} module:

\begin{prompt}
%source \$FOAM\_BASH%
\end{prompt}


\subsubsection{Defining utility functions used in tutorial cases}

If you would like to use the \texttt{\small{getApplication}}, \texttt{\small{runApplication}},
\texttt{\small{runParallel}}, \texttt{\small{cloneCase}} and/or \texttt{\small{compileApplication}} functions that are
used in OpenFOAM tutorials, you also need to \texttt{\small{source}} the \texttt{\small{RunFunctions}} script:

\begin{prompt}
%source \$WM\_PROJECT\_DIR/bin/tools/RunFunctions%
\end{prompt}

Note that this needs to be done \textbf{after} sourcing \texttt{\small{\$FOAM\_BASH}} to make sure
\texttt{\small{\$WM\_PROJECT\_DIR}} is defined.


\subsubsection{Dealing with floating-point errors}

If you are seeing \texttt{\small{Floating Point Exception}} errors, you can undefine the
\texttt{\small{\$FOAM\_SIGFPE}} environment variable that is defined by the \texttt{\small{\$FOAM\_BASH}} script,
as follows:

\begin{prompt}
%unset \$FOAM\_SIGFPE%
\end{prompt}

Note that this only prevents OpenFOAM from propogating these floating point exceptions which then results in
terminating the simulation; it does not prevent that illegal operations (like division-by-zero) are being executed.
As such, \textbf{you should \textit{not} use this in production runs}. Instead, you should track down the root cause
of the floating point exceptions, and try to prevent them from occuring at all.


\subsection{Running OpenFOAM in parallel}

Note: When running OpenFOAM in parallel, \textbf{do not forget to specify the \texttt{\small{-parallel}} option},
to avoid running the same OpenFOAM simulation $N$ times, rather than running it once using $N$ processor cores.

\subsubsection{Using \texttt{mympirun}}

When running OpenFOAM in parallel, it is highly recommended to use the \texttt{mympirun} command rather than
the standard \texttt{mpirun} command;
see Section~\ref{sec:best-practices-mympirun} for more information on \texttt{mympirun}.

To use \texttt{mympirun}, make sure that the \texttt{\small{vsc-mympirun}} module is loaded.

\begin{prompt}
%module load vsc-mympirun%
\end{prompt}

Note that you should \textit{not} specify a specific version here, see also Section~\ref{sec:best-practices-mympirun-module}.

Whenever you are instructed to use a command like `\texttt{\small{mpirun -np <N>}} ...',
use `\texttt{\small{mympirun}} ...' instead; \texttt{\small{mympirun}} will automatically detect the number of
processor cores that are available (see also Section~\ref{}).

\subsubsection{Passing down OpenFOAM environment variables to MPI processes}

To pass down the environment variables required to run OpenFOAM (which were defined by the
\texttt{\small{\$FOAM\_BASH}} script, see Section~\ref{sec:best-practices-openfoam-environment})
to each of the MPI processes used in a parallel OpenFOAM execution,
define the \texttt{\small{\$MYMPIRUN\_VARIABLESPREFIX}} environment variable as follows,
prior to running the OpenFOAM simulation with \texttt{\small{mympirun}}:

\begin{prompt}
%export MYMPIRUN\_VARIABLESPREFIX=WM\_PROJECT,FOAM,MPI%
\end{prompt}

Note: this is equivalent to using `\texttt{\small{mympirun --variablesprefix=WM\_PROJECT,FOAM,MPI ...}}'.


\subsection{Running OpenFOAM on a shared filesystem}

\begin{itemize}
\item zipped
\item avoid dumping data for *every* time step
\end{itemize}

\subsection{OpenFOAM workflow}

\begin{itemize}
\item meshing (\texttt{decomposePar}): as a part of job script vs up front, attention points (e.g., \# cores used in job)
\item running simulation
\item post-processing (e.g.\ \texttt{reconstructPar}, ParaView)
\end{itemize}

\subsection{Scaling of OpenFOAM on VSC HPC clusters}

\begin{itemize}
\item scaling on relevant (HPC-UGent) Tier-2 systems
\item scaling on Tier-1b
\end{itemize}

using which input(s)?

\subsection{Using own solvers with OpenFOAM}

\begin{itemize}
\item compiling own solvers
\item using own solvers in jobs
\end{itemize}

\subsection{Example OpenFOAM job script}

\verbatiminput{examples/Best-practices/OpenFOAM/example.sh}
