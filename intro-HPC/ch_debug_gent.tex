\chapter{HPC-UGent debug cluster}
\label{ch:debug_ugent}


\section{Purpose}
\label{sec:debug_ugent_pupose}
The purpose of the debug cluster is to run debug jobs. Additionally, it can be also used for courses. 
The debug cluster is overcommited, which means that more CPU cores can be
assigned to jobs than physically exist in the cluster. Obviously, the performance of this cluster
heavily depends on the workloads and that actual overcommit usage. Be aware that jobs can slow
down or speed up during their execution.
\\ Due to the restrictions and sharing the CPU resources (see section~\ref{subsec:debug_ugent_restrictions}) 
jobs on slaking should normally start more or less immediately. As a consequence,
typical workloads for this cluster should be limited to:
\begin{itemize}
  \item  Debugging programs
  \item  Try/debug submit scripts
  \item  Run small jobs for which performance is not an issue (courses, hands-ons, etc.)
  \item  Run interactive jobs (see chapter~\ref{ch:running-interactive-jobs}) with lots of idle time
         (courses, hands-ons, etc.) 
  \item  Run Cluster Desktops (see chapter~\ref{ch:web_portal}) with lots of idle time
         (courses, hands-ons, etc.)
\end{itemize} 

\section{Submitting jobs}
\label{sec:debug_ugent_jobs}

To submit jobs to the HPC-UGent debug cluster nicknamed \lstinline|slaking|, first use:

\begin{prompt}
%\shellcmd{module swap cluster/slaking}
\end{prompt}

Then use the familiar \lstinline|qsub|, \lstinline|qstat|, etc. commands.

\subsection{Restrictions and overcommit factor}
\label{subsec:debug_ugent_restrictions}

Currently each user may have at most 5 jobs in the queue (both running and waiting to run)
and at most 3 jobs per user can be running at the same time. The running jobs may allocate 
no more than 8 CPU cores in total. \\
At this moment, the cluster has an overcommit factor 6.  This means that 6 times more cores
can be allocated than physically exist. Simulatenously, the default memory per core is 6
times less than what would physically be available on a non-overcommitted cluster. \\
Please note that based on the (historical) workload of the debug cluster, the above restrictions
and the overcommitment ratio might change without prior notice.
