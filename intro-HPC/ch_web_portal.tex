\chapter{Using the HPC-UGent web portal}
\label{ch:web_portal}

The HPC-UGent web portal provides ``one stop shop'' for the HPC-UGent infrastructure.
It is based on \href{https://openondemand.org/}{\strong{Open OnDemand}} (or \texttt{OoD} for short).

Via this web portal you can upload and download files, create, edit, submit, and monitor jobs,
run GUI applications, and connect via SSH, all via a standard web browser like Firefox, Chrome
or Safari. You do not need to install or configure any client software,
and no SSH key is required to connect to your VSC account via this web portal.
Please note that we do recommend to use our interactive cluster
(see chapter ~\ref{ch:interactive_ugent}) with \texttt{OoD}

To connect to the HPC-UGent infrastructure via the web portal, visit:

\begin{center}\url{https://login.hpc.ugent.be}\end{center}

Note that you may only see a ``\emph{Submitting...}'' message appear for a couple of seconds,
which is perfectly normal.

Through this web portal, you can:
\begin{itemize}
    \item browse through the files \& directories in your VSC account, and inspect, manage or change them;
    \item consult active jobs (across all HPC-UGent Tier-2 clusters);
    \item submit new jobs to the HPC-UGent Tier-2 clusters, either from existing job scripts or from job templates;
    \item start an interactive graphical user interface (a desktop environment), either on the login nodes or on a cluster workernode;
    \item open a terminal session directly in your web browser;
\end{itemize}

More detailed information is available below, as well as in the \href{https://osc.github.io/ood-documentation/master/}{Open OnDemand documentation}.
A walkthrough video is available on YouTube \href{https://www.youtube.com/watch?v=4-w-4wjlnPk}{here}.

\section*{Pilot access}

\strong{The HPC-UGent web portal is currently available for all VSC accounts,
but there may be a couple of small problems that we still need to be resolved.}

\strong{If you notice anything that doesn't work as expected or could be improved,
please contact \hpcinfo.}

\strong{In addition, during the pilot phase you \emph{must} be connected to the HPC-UGent network
(via \href{https://helpdesk.ugent.be/vpn/en/}{VPN} if necessary)
in order to use the HPC-UGent web portal.}

\subsection{Known issues \& limitations}

\paragraph{Limited resources}

All web portal sessions are currently served through a single separate login node,
so the available resources are relatively limited.
We will monitor the resources used by the active web portal sessions throughout the pilot
phase to evaluate whether more resources are required.

\section{Login}

When visiting the HPC-UGent web portal you will be automatically logged in via the VSC accountpage
(see also Section~\ref{sec:applying-for-the-account}).

\subsection{First login}

The first time you visit \url{https://login.hpc.ugent.be} permission will be requested to let the web portal 
access some of your personal information (VSC login ID, account status, login shell and institute name),
as shown in this screenshot below:

\begin{center}
    \includegraphics*[scale=0.5]{ood_permission}
\end{center}

\strong{Please click "Authorize" here.}

This request will only be made once, you should not see this again afterwards.

\section{Start page}

Once logged in, you should see this start page:

\begin{center}
    \includegraphics*[scale=0.4]{ood_start}
\end{center}

This page includes a menu bar at the top, with buttons on the left providing access to the different features supported
by the web portal, as well as a \emph{Help} menu, your VSC account name, and a \emph{Log Out} button on the top right,
and the familiar HPC-UGent welcome message with a high-level overview of the HPC-UGent Tier-2 clusters.

If your browser window is too narrow, the menu is available at the top right through the ``hamburger'' icon:

\begin{center}
    \includegraphics*[scale=0.4]{ood_hamburger}
\end{center}


\section{Features}

We briefly cover the different features provided by the web portal, going from left to right
in the menu bar at the top.

\subsection{File browser}

Via the \emph{Files} drop-down menu at the top left, you can browse through the files and directories in your
VSC account using an intuitive interface that is similar to a local file browser, and manage, inspect or change them.

The drop-down menu provides
short-cuts to the different \lstinline|$VSC_*| directories and filesystems you have access to.
Selecting one of the directories will open a new browser tab with the \emph{File Explorer}:

\begin{center}
    \includegraphics*[scale=0.5]{ood_file_explorer}
\end{center}

Here you can:

\begin{itemize}
    \item Click a directory in the tree view on the left to open it;
    \item Use the buttons on the top to:
    \begin{itemize}
        \item go to a specific subdirectory by typing in the path (via \emph{Go To...});
        \item open the current directory in a terminal (shell) session (via \emph{Open in Terminal});
        \item create a new file (via \emph{New File}) or subdirectory (via \emph{New Dir}) in the current directory;
        \item upload files or directories from your local workstation into your VSC account, in the currect directory (via \emph{Upload});
        \item show hidden files and directories, of which the name starts with a dot (\lstinline|.|) (via \emph{Show Dotfiles});
        \item show the owner and permissions in the file listing (via \emph{Show Owner/Mode});
    \end{itemize}
    \item Double-click a directory in the file listing to open that directory;
    \item Select one or more files and/or directories in the file listing, and:
    \begin{itemize}
        \item use the \emph{View} button to see the contents (use the button at the top right to close the resulting popup window);
        \item use the \emph{Edit} button to open a simple file editor in a new browser tab which you can use to make changes to the selected file and save them;
        \item use the \emph{Rename/Move} button to rename or move the selected files and/or directories to a different location in your VSC account;
        \item use the \emph{Download} button to download the selected files and directories from your VSC account to your local workstation;
        \item use the \emph{Copy} button to copy the selected files and/or directories, and then use the \emph{Paste} button to paste them in a different location;
        \item use the \emph{(Un)Select All} button to select (or unselect) all files and directories in the current directory;
        \item use the \emph{Delete} button to (\strong{permanently!}) remove the selected files and directories;
    \end{itemize}
\end{itemize}

For more information, see aslo \url{https://www.osc.edu/resources/online_portals/ondemand/file_transfer_and_management}.


\subsection{Job management}

Via the \emph{Jobs} menu item, you can consult your active jobs or submit new jobs using the \emph{Job Composer}.

For more information, see the sections below as well as
\url{https://www.osc.edu/resources/online_portals/ondemand/job_management}.

\subsubsection{Active jobs}

To get an overview of all your currently active jobs,
use the \emph{Active Jobs} menu item under \emph{Jobs}.

A new browser tab will be opened that shows all your current queued and/or running jobs:

\begin{center}
    \includegraphics*[scale=0.5]{ood_active_jobs}
\end{center}

You can control which jobs are shown using the \emph{Filter} input area,
or select a particular cluster from the drop-down menu \emph{All Clusters}, both at the top right.

Jobs that are still queued or running can be deleted using the red button on the right.

Completed jobs will also be visible in this interface, but only for a short amount of time after they have stopped running.

For each listed job, you can click on the arrow ($>$) symbol to get a detailed overview of that job,
and get quick access to the corresponding output directory (via the \emph{Open in File Manager} and \emph{Open in Terminal} buttons at the bottom of the detailed overview).

\subsubsection{Job composer}

To submit new jobs, you can use the \emph{Job Composer} menu item under \emph{Jobs}.
This will open a new browser tab providing an interface to create new jobs:

\begin{center}
    \includegraphics*[scale=0.35]{ood_job_composer}
\end{center}

This extensive interface allows you to create jobs from one of the available templates,
or by copying an existing job.

You can carefuly prepare your job and the corresponding job script via the \emph{Job Options} button
and by editing the job script (see lower right).

Don't forget to actually submit your job to the system via the green \emph{Submit} button!

\paragraph{Job templates}

In addition, you can inspect provided job templates, copy them or even create your own templates via the \emph{Templates} button on the top:

\begin{center}
    \includegraphics*[scale=0.35]{ood_job_templates}
\end{center}

\subsection{Shell access}

Through the \emph{Shell Access} button that is available under the \emph{Clusters} menu item,
you can easily open a terminal (shell) session into your VSC account, straight from your browser!

\begin{center}
    \includegraphics*[scale=0.5]{ood_shell}
\end{center}

Using this interface requires being familiar with a Linux shell environment (see Appendix~\ref{ch:useful-linux-commands}).

To exit the shell session, type \lstinline|exit| followed by \emph{Enter} and then close the browser tab.

Note that you can not access a shell session after you closed a browser tab, even if you didn't exit the
shell session first (unless you use terminal multiplexer tool like \lstinline|screen| or \lstinline|tmux|).

\subsection{Interactive applications}

\subsubsection{Graphical desktop environment}

To create a graphical desktop environment, use on of the \emph{desktop on ... node} buttons under \emph{Interactive Apps} menu item. For example:

\begin{center}
    \includegraphics*[scale=0.35]{ood_launch_desktop}
\end{center}

You can either start a desktop environment on a login node
for some lightweight tasks, or on a workernode of one of the HPC-UGent Tier-2 clusters if
more resources are required.
Keep in mind that for desktop sessions on a workernode 
the regular queueing times are applicable dependent on requested resources.

\strong{Do keep in mind that desktop environments on a cluster workernode are limited to a maximum of
72 hours, just like regular jobs are.}

To access the desktop environment, click the \emph{My Interactive Sessions} menu item at the top,
and then use the \emph{Launch desktop on ... node} button if the desktop session is \emph{Running}:

\begin{center}
    \includegraphics*[scale=0.40]{ood_desktop_running}
\end{center}

\subsubsection{Jupyter notebook}

Through the web portal you can easily start a \href{https://jupyter.org/}{Jupyter notebook}
on a workernode, via the \emph{Jupyter Notebook} button under the \emph{Interactive Apps} menu item.

\begin{center}
    \includegraphics*[scale=0.4]{ood_start_jupyter}
\end{center}

After starting the Jupyter notebook using the \emph{Launch} button,
you will see it being added in state \emph{Queued} in the overview of interactive sessions
(see \emph{My Interactive Sessions} menu item):


\begin{center}
    \includegraphics*[scale=0.45]{ood_jupyter_queued}
\end{center}

When your job hosting the Jupyter notebook starts running, the status will first change the \emph{Starting}:

\begin{center}
    \includegraphics*[scale=0.45]{ood_jupyter_starting}
\end{center}

and eventually the status will change to \emph{Running}, and you will be able to connect
to the Jupyter environment using the blue \emph{Connect to Jupyter} button:

\begin{center}
    \includegraphics*[scale=0.45]{ood_jupyter_running}
\end{center}

This will launch the Jupyter environment in a new browser tab, where you can open
an existing notebook by navigating to the directory where it located and clicking it,
or using the \emph{New} menu on the top right:

\begin{center}
    \includegraphics*[scale=0.6]{ood_jupyter_new_notebook}
\end{center}

Here's an example of a Jupyter notebook in action. Note that several non-standard Python packages
(like \emph{numpy}, \emph{scipy}, \emph{pandas}, \emph{matplotlib}) are readily available:

\begin{center}
    \includegraphics*[scale=0.5]{ood_jupyter_notebook_example}
\end{center}


\section{Restarting your web server in case of problems}

In case of problems with the web portal, it could help to restart the web server running in your VSC account.

You can do this via the \emph{Restart Web Server} button under the \emph{Help} menu item:

\begin{center}
    \includegraphics*[scale=0.4]{ood_help_restart_web_server}
\end{center}

Of course, this only affects your own web portal session (not those of others).
