\chapter{Job script examples}
\label{ch:jobscript-examples}

\section{Single-core job}

Here's an example of a single-core job script:

\examplecode{bash}{single_core.sh}

\begin{enumerate}
    \item We set \lstinline|#PBS| headers, see \autoref{ch:torque-options} for a list of these options
    \item We load a module, see \autoref{sec:modules}
    \item We stage the data in
    \item We execute the main part of the job
    \item We stage the results out. For a list of possible storage locations, see \autoref{subsec:predefined-user-directories}.
\end{enumerate}

\section{Multi-core job}

Here's an example of a multi-core job script that uses \lstinline|mympirun|:

\examplecode{bash}{multi_core.sh}

An example MPI hello world program can be downloaded from \url{https://www.open-mpi.org/papers/workshop-2006/hello.c}.


\section{Running a command with a maximum time limit}
\label{sec:maximum-timelimit-timeout-jobscript}

If you want to run a job, but you are not sure it will finish before the job runs
out of walltime and we want to copy data back before, you have to stop the main
command before the walltime runs out and copy the data back.

This can be done with the \lstinline|timeout| command. This command sets a limit
of time a program can run for, and when this limit is exceeded, it kills the program.
Here's an example job script using \lstinline|timeout|:

\examplecode{bash}{timeout.sh}

The example program used in this script is a dummy script that simply sleeps a specified amount of minutes:

\examplecode{bash}{example_program.sh}
