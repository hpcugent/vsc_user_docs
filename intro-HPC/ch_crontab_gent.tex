\chapter{Cron scripts}
\label{ch:crontab_ugent}


\section{Cron scripts configuration}
\label{sec:crontab_ugent_cron_scripts}
It is possible to run automated cron scripts as regular user on the Ugent login nodes.
Due to the high availability setup users should add their cron scripts on the same login
node to avoid any cron job script duplication.

In order to create a new cron script first login to HPC-UGent login node as
usual with your vsc user's account (see section \autoref{ch:connecting}).

Check if any cron script is already set in the current login node with:


\begin{prompt}
%\shellcmd{crontab -l}
\end{prompt}

At this point you can add/edit (with \lstinline|vi| editor) any cron script running the command:

\begin{prompt}
%\shellcmd{crontab -e}
\end{prompt}

\subsubsection{Example cron job script}
\label{sec:crontab_ugent_cron_script_example}
\begin{prompt}
 15 5 * * * ~/runscript.sh >& ~/job.out
\end{prompt}

where \lstinline|runscript.sh| has these lines in this example:

\examplecode{bash}{runscript.sh}

In the previous example a cron script was set to be executed every day at 5:15 am.
More information about crontab and cron scheduling format at 
\url{https://www.redhat.com/sysadmin/automate-linux-tasks-cron}.

Please note that you should login into the same login node to edit your previously 
generated crontab tasks. If that is not the case you can always jump from one login node
to another with:

\begin{prompt}
%\shellcmd{ssh gligar<id>}
\end{prompt}




