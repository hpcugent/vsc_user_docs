\chapter{Frequently Asked Questions}
\label{ch:faq}

\section{When will my job start?}

\ifgent
See the explanation about how jobs get prioritized in \autoref{subsec:priority}.
\else
\ifbrussel
See the explanation about how jobs get prioritized in \autoref{subsec:priority}.
\else
You can use the \lstinline|showstart| command. For more information, see \autoref{sec:monitoring-and-managing-your-jobs}.
\fi % BRUSSEL
\fi % GENT

\section{Can I share my account with someone else?}

\strong{NO.} You are not allowed to share your VSC account with anyone else, it is strictly personal.
\ifgent
See \url{https://helpdesk.ugent.be/account/en/regels.php}.
\fi
\ifleuven
For KU Leuven, see \url{https://admin.kuleuven.be/personeel/english_hrdepartment/ICT-codeofconduct-staff#section-5}.
For Hasselt University, see \url{https://www.uhasselt.be/intra/IVC}.
\fi
\ifbrussel
See \url{http://www.vub.ac.be/sites/vub/files/reglement-gebruik-ict-infrastructuur.pdf}.
\fi
\ifantwerpen
See \url{https://pintra.uantwerpen.be/bbcswebdav/xid-23610_1}
\fi
\ifgent
If you want to share data, there are alternatives (like a shared
directories in VO space, see \autoref{sec:virtual-organisations}).
\fi

\section{Can I share my data with other \hpc users?}

Yes, you can use the \lstinline|chmod| or \lstinline|setfacl| commands to change permissions
of files so other users can access the data. For example, the following command
will enable a user named ``otheruser'' to read the file named \lstinline|dataset.txt|.
See

\begin{prompt}
%\shellcmd{setfacl -m u:otheruser:r dataset.txt}%
%\shellcmd{ls -l dataset.txt}%
-rwxr-x---+ 2 %\userid{}% mygroup      40 Apr 12 15:00 dataset.txt
\end{prompt}

For more information about \lstinline|chmod| or \lstinline|setfacl|, see \href{\LinuxManualURL#sec:chmod}
{the section on chmod in chapter 3 of the Linux intro manual}.
% \section{I no longer work for \university, can I transfer my data to another researcher working at \university}
% See https://github.com/hpcugent/vsc_user_docs/issues/230

\section{Can I use multple different SSH key pairs to connect ot my VSC account?}

Yes, and this is recommendend when working from different computers. Please see
\autoref{sec:adding-multiple-keys} on how to do this.

\section{I want to use software that is not available on the clusters yet}

\ifgent
Please fill out the details about the software and why you need it in this form:
\url{https://www.ugent.be/hpc/en/support/software-installation-request}.
When submitting the form, a mail will be sent to \hpcinfo containing all the
provided information. The HPC team will look into your request as soon as possible
you and contact you when the installation is done or if further information is required.
\else
Please send an e-mail to \hpcinfo that includes:
\begin{itemize}
    \item What software you want to install and the required version
    \item Detailed installation instructions
    \item The purpose for which you want to install the software
\end{itemize}

\ifleuven
Alternatively, you can also fill out the details about the software and why you need it in this form:
\url{https://admin.kuleuven.be/icts/HPCinfo_form/HPC-info-formulier}.
When submitting the form, a mail will be sent to \hpcinfo containing all the
provided information. The HPC team will look into your request as soon as possible
you and contact you when the installation is done or if further information is required.
\fi

\ifgent
\section{How do I get access to the GPGPU system in Leuven?}

The HPC at the KU Leuven has got several GPGPU nodes.

You have to be a member of the \lstinline|lp_GPU_GENT| group before you can
access these GPU nodes. See \autoref{subsec:joining-existing-group} for how to join that group.

To use the GPGPU nodes, log in (use the same public key as you use for the UGent HPC):

\begin{prompt}
%\shellcmd{ssh \userid{}@login.hpc.kuleuven.be}%
\end{prompt}

For example, to submit a job script called \lstinline|example_jobscript.pbs|:

\begin{prompt}
qsub -lpartition=gpu -lnodes=1:gpu -A lp_GPU_GENT example_jobscript.pbs
\end{prompt}

The last argument (\lstinline|-A lp_GPU_GENT|) is needed to have sufficient credits
(the KU Leuven works with a credit system).

Submitting to either GPU or Phi clusters is possible with: \lstinline|qsub -lpartition=gpu|
or \lstinline|qsub -lpartition=phi|.

Selecting a specific gpu node is also possible:
\lstinline|qsub -lpartition=gpu,nodes=1:M2070| or \lstinline|qsub -lpartition=gpu,nodes=1:K20Xm|

In your job script, use a login shell. For bash, this means the first line of your jobscript should be:
\lstinline|#!/bin/bash -l|.

Don't forget to load \lstinline|cuda|: \lstinline|module load cuda|.

Additional information can be found at \url{https://vscentrum.be/neutral/documentation/cluster-doc/running-jobs/job-system-basics}

The KULeuven Helpdesk can be contacted at \lstinline|hpcinfo@icts.kuleuven.be|.


The regular \lstinline|$VSC_HOME| and \lstinline|$VSC_DATA| can also be accessed,
but the path's differ from the UGent login nodes. Always use the bash variables \lstinline|$VSC_HOME|
and \lstinline|$VSC_DATA|. They will point to the correct location.

More information is available at \url{https://vscentrum.be/neutral/documentation/support/tut-book/vsc-tutorials/thinking-quick-start-guide.pdf}.
\fi
