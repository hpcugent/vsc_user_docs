\chapter{Navigating}

\section{Current directory: ``pwd'' and ``\$PWD''}

To print the current directory, use \verb|pwd| or \verb|$PWD|:

\begin{prompt}
%\shellcmd{cd \$HOME}%
%\shellcmd{pwd}%
%\homedir%
%\shellcmd{echo "The current directory is: \$PWD"}%
The current directory is: %\homedir%
\end{prompt}

\section{Listing files and directories: ``ls''}

A very basic and commonly used command is \verb|ls|, which can be used to list files and directories.

In it's basic usage, it just prints the names of files and directories in the current directory. For example:

\begin{prompt}
%\shellcmd{ls}%
afile.txt   some_directory
\end{prompt}

When provided an argument, it can be used to list the contents of a directory:

\begin{prompt}
%\shellcmd{ls some\_directory}%
one.txt  two.txt
\end{prompt}

A couple of commonly used options include:

\begin{itemize}
\item detailed listing using \verb|ls -l|:

\begin{prompt}
%\shellcmd{ls -l}%
total 4224
-rw-rw-r-- 1 %\userid% %\userid% 2157404 Apr 12 13:17 afile.txt
drwxrwxr-x 2 %\userid% %\userid%     512 Apr 12 12:51 some_directory
\end{prompt}

\item To print the size information in human-readable form, use the \verb|-h| flag:

\begin{prompt}
%\shellcmd{ls -lh}%
total 4.1M
-rw-rw-r-- 1 %\userid% %\userid% 2.1M Apr 12 13:16 afile.txt
drwxrwxr-x 2 %\userid% %\userid%  512 Apr 12 12:51 some_directory
\end{prompt}

\item also listing hidden files using the \verb|-a| flag:

\begin{prompt}
%\shellcmd{ls -lah}%
total 3.9M
drwxrwxr-x   3 %\userid% %\userid%  512 Apr 12 13:11 .
drwx------ 188 %\userid% %\userid% 128K Apr 12 12:41 ..
-rw-rw-r--   1 %\userid% %\userid% 1.8M Apr 12 13:12 afile.txt
-rw-rw-r--   1 %\userid% %\userid%    0 Apr 12 13:11 .hidden_file.txt
drwxrwxr-x   2 %\userid% %\userid%  512 Apr 12 12:51 some_directory
\end{prompt}

\item ordering files by the most recent change using \verb|-rt|:

\begin{prompt}
%\shellcmd{ls -lrth}%
total 4.0M
drwxrwxr-x 2 %\userid% %\userid%  512 Apr 12 12:51 some_directory
-rw-rw-r-- 1 %\userid% %\userid% 2.0M Apr 12 13:15 afile.txt
\end{prompt}

\end{itemize}

If you try to use \verb|ls| on a file that doesn't exist, you will get a clear error message:

\begin{prompt}
%\shellcmd{ls nosuchfile}%
ls: cannot access nosuchfile: No such file or directory
\end{prompt}

\section{Changing directory: ``cd''}

To change to a different directory, you can use the \verb|cd| command:

\begin{prompt}
%\shellcmd{cd some\_directory}%
\end{prompt}

To change back to the previous directory you were in, there's a shortcut: \verb|cd -|

Using \verb|cd| without an argument results in returning back to your home directory:

\begin{prompt}
%\shellcmd{cd}%
%\shellcmd{pwd}%
%\homedir%
\end{prompt}

\section{Inspecting file type: ``file''}

The \verb|file| command can be used to inspect what type of file you're dealing with:

\begin{prompt}
%\shellcmd{file afile.txt}%
afile.txt: ASCII text

%\shellcmd{file some\_directory}%
some_directory: directory
\end{prompt}

\section{Absolute vs relative file paths}

An \emph{absolute} filepath starts with \verb|/| (or a variable which value starts
with  \verb|/|), which is also called the \emph{root} of the filesystem.

Example: absolute path to your home directory:
\texttt{\homedir}.

A \emph{relative} path starts from the current directory, and points to another
location up or down the filesystem hierarchy.

Example: \verb|some_directory/one.txt| points to the file \verb|one.txt| that is
located in the subdirectory named \verb|some_directory| of the current directory.

There are two special relative paths worth mentioning:

\begin{itemize}
    \item \verb|.| is a shorthand for the current directory
    \item \verb|..| is a shorthand for the parent of the current directory
\end{itemize}

You can also use \verb|..| when constructing relative paths, for example:

\begin{prompt}
%\shellcmd{cd \$HOME/some\_directory}%
%\shellcmd{ls ../afile.txt}%
../afile.txt
\end{prompt}

\section{Permissions}
\label{sec:permissions}

Each file and directory has particular \emph{permissions} set on it, which can be queried using \verb|ls -l|.

For example:

\begin{prompt}
%\shellcmd{ls -l afile.txt}%
-rw-rw-r-- 1 %\userid% agroup 2929176 Apr 12 13:29 afile.txt
\end{prompt}

The \verb|-rwxrw-r--| specifies both the type of file (\verb|-| for files, \verb|d|
for directories (see first character)), and the permissions for user/group/others:

\begin{enumerate}
\item each triple of characters indicates whether the read (\verb|r|), write (\verb|w|), execute (\verb|x|) permission bits are set or not
\item the 1st part \verb|rwx| indicates that the \emph{owner} ``\userid'' of the file has all the rights
\item the 2nd part \verb|rw-| indicates the members of the \emph{group} ``agroup'' only have read/write permissions (not execute)
\item the 3rd part \verb|r--| indicates that \emph{other} users only have read permissions
\end{enumerate}

The default permission settings for new files/directories are determined by the so-called \emph{umask} setting, and are by default:

\begin{enumerate}
\item read-write permission on files for user/group (no execute), read-only for others (no write/execute)
\item read-write-execute permission for directories on user/group, read/execute-only for others (no write)
\end{enumerate}

See also \hyperref[sec:chmod]{the chmod command} later in this manual.

\section{Finding files/directories: ``find''}

\verb|find| will crawl a series of directories and lists files matching given criteria.

For example, to look for the file named \verb|one.txt|:

\begin{prompt}
%\shellcmd{cd \$HOME}%
%\shellcmd{find . -name one.txt}%
./some_directory/one.txt
\end{prompt}

To look for files using incomplete names, you can use a wildcard \verb|*|; note
that you need to escape the \verb|*| to avoid that Bash \emph{expands} it into
\verb|afile.txt| by adding double quotes:

\begin{prompt}
%\shellcmd{find . -name "*.txt"}%
./.hidden_file.txt
./afile.txt
./some_directory/one.txt
./some_directory/two.txt
\end{prompt}

A more advanced use of the \verb|find| command is to use the \verb|-exec| flag
to perform actions on the found file(s), rather than just printing their paths (see \verb|man find|).
