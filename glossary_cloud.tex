\newglossaryentry{OpenStack}
{ name={OpenStack},
  description={OpenStack (\url{https://www.openstack.org}) is a free
    and open-source software platform for cloud computing, mostly
    deployed as infrastructure-as-a-service (IaaS), whereby virtual
    servers and other resources are made available to customers}, }
\newglossaryentry{OpenStack Identity}
{ name={OpenStack Identity},
  description={OpenStack Identity (Keystone) provides a central
    directory of users mapped to the OpenStack services they can
    access. It acts as a common authentication system across the cloud
    operating system and can integrate with existing backend directory
    services. It supports multiple forms of authentication including
    standard username/password credentials and token-based systems.
    In the VSC cloud, it is integrated with the VSC account system.},
}
\newglossaryentry{OpenStack Image}
{
  name={OpenStack Image},
  description={OpenStack Image (Glance) provides discovery,
  registration, and delivery services for disk and server images. Stored
  images can be used as a template. It can also be used to store and
  catalog an unlimited number of backups. The Image Service can store
  disk and server images in a variety of back-ends, including Swift.},
}
\newglossaryentry{OpenStack Dashboard}
{
  name={OpenStack Dashboard},
  description={OpenStack Dashboard (Horizon) provides administrators and
  users with a graphical interface to access, provision, and automate
  deployment of cloud-based resources. The design accommodates third
  party products and services, such as billing, monitoring, and
  additional management tools. The dashboard is also brand-able for
  service providers and other commercial vendors who want to make use of
  it. The dashboard is one of several ways users can interact with
  OpenStack resources. Developers can automate access or build tools to
  manage resources using the native OpenStack API or the EC2
  compatibility API.},
}
\newglossaryentry{Horizon}
{
  name={Horizon},
  description={Horizon is the name of the \gls{OpenStack Dashboard}.},
}
\newglossaryentry{Heat}
{
  name={Heat},
  description={Heat is the OpenStack orchestration service, which can
  manage multiple composite cloud applications using templates,
  through both an OpenStack-native \gls{REST} API and a
  CloudFormation-compatible Query API},
}
\newglossaryentry{REST}
{
  name={\textsc{rest}},
  description={REpresentational State Transfer is a software
  architectural style that defines a set of constraints to be used for
  creating web services},
}
\newglossaryentry{stack}
{
  name={stack},
  description={In the context of \gls{OpenStack}, a stack is a
  collection of cloud resources which can be managed using the
  \gls{Heat} orchestration engine},
}
\newglossaryentry{Heat Orchestration Template}
{
  name={Heat Orchestration Template},
  description={A \gls{Heat} Orchestration Template (\textsc{hot}) is a text
  file which describes the infrastructure for a cloud application.
  Because \textsc{hot} files are text files in a \gls{yaml}-based format, they
  are readable and writable by humans, and can be managed using a
  version control system.  \textsc{hot} is one of the template formats
  supported by Heat, along with the older CloudFormation-compatible
  \textsc{cfn} format},
}
\newglossaryentry{instance}
{
  name={OpenStack Instance},
  description={OpenStack Instances are virtual machines, which are
  instances of a system image that is created upon request and which
  is configured when launched. With traditional virtualization
  technology, the state of the virtual machine is persistent, whereas
  OpenStack supports both persistent and ephemeral image creation},
}
\newglossaryentry{OpenStack Volume}
{
  name={OpenStack Volume},
  description={OpenStack Volume is a detachable block storage device
  and each volume can be attached to only one instance},
}
\newglossaryentry{yaml}
{
  name={\textsc{yaml}},
  description={A human-readable text-based data serialization
  format}
}
\newglossaryentry{nfs}
{
  name={\textsc{nfs}},
  description={Network File System, a protocol for sharing file
    systems across a network, often used on Unix(-like) operating
    systems.}
}
\newglossaryentry{nic}
{
  name={\textsc{nic}},
  description={Network Interface Controller, a (virtualized)
    hardware component that connects a computer to a network.}
}
%%% Local Variables:
%%% mode: latex
%%% TeX-master: "intro-OpenStack/intro-OpenStack"
%%% End:
