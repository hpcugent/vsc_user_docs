% use \macro instead of \newcommand so we automatically add a space
\usepackage{xspace}
\newcommand{\macro}[2]{\newcommand{#1}{#2\xspace}}
\ifgent
% GENT SPECIFIC MACROS
\macro{\sitename}{gent}
\macro{\hpc}{HPC-UGent}
\macro{\jobid}{123456.master15.delcatty.gent.vsc}
\macro{\loginnode}{gengar.ugent.be}
\macro{\userid}{vsc40000}
\macro{\university}{UGent}
\macro{\city}{Gent}
\macro{\operatingsystem}{Scientifc Linux 6.5}

\fi
\ifleuven
% LEUVEN SPECIFIC MACROS
\macro{\sitename}{leuven}
\macro{\hpc}{HPC-KU Leuven/UHasselt}
\macro{\jobid}{123456.svcs.hpc.leuven.vsc}
\macro{\loginnode}{login.hpc.kuleuven.be}
\macro{\userid}{vsc30000}
\macro{\university}{KU Leuven/Hasselt University}
\macro{\association}{KU Leuven/Hasselt University and their association partners}
\macro{\city}{Leuven}
\macro{\operatingsystem}{Redhat Enterprise Linux 6.3}

\fi
\ifhasselt
% HASSELT SPECIFIC MACROS

\fi
\ifantwerpen
% ANTWERPEN SPECIFIC MACROS
\macro{\sitename}{antwerpen}
\macro{\hpc}{UA-HPC}
\macro{\jobid}{433253.master1.turing.antwerpen.vsc}
\macro{\userid}{vsc20167}
\macro{\loginnode}{login.turing.calcua.ua.ac.be}
\macro{\university}{UA}
\macro{\association}{Antwerp University Association (AUHA)}
\macro{\city}{Antwerp}
\macro{\operatingsystem}{Scientifc Linux 5.4}

\fi
\ifbrussel
% BRUSSEL SPECIFIC MACROS

\fi
