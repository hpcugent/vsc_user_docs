% use \macro instead of \newcommand so we automatically add a space
\usepackage{xspace}

\newcounter{cnt}
\newcommand\textlist{}
\newcommand\settext[2]{%
\csdef{text#1}{#2}}
\newcommand\addtext[1]{%
  \stepcounter{cnt}%
\csdef{text\thecnt}{#1}}
\newcommand\gettext[1]{%
\csuse{text#1}}

\newcounter{colnum}
\newcommand\maketabularrow[1]{%
  \setcounter{colnum}{0}%
  \whileboolexpr
  { test {\ifnumcomp{\value{colnum}}{<}{#1 - 1}} }%
  {\stepcounter{colnum}\gettext{\thecolnum} & \gettext{\gettext{\thecolnum}} \\ \hline }
  % Weird stuff with latex adding an empty row if we don't use the following way.
  \stepcounter{colnum}\gettext{\thecolnum} & \gettext{\gettext{\thecolnum}}
}

\newcommand{\macro}[2]{\addtext{\string#1}\settext{\string#1}{\noexpand#2}\newcommand{#1}{#2\xspace}}

% Define some general macros
\ifwindows
\macro{\OS}{Windows}
\newcommand{\osurl}{windows}
\fi
\ifmac
\macro{\OS}{Mac}
\newcommand{\osurl}{mac}
\fi
\iflinux
\macro{\OS}{Linux}
\newcommand{\osurl}{linux}
\fi

\ifleuven
  \newcommand{\siteurl}{leuven}
\fi
\ifantwerpen
  \newcommand{\siteurl}{antwerpen}
\fi
\ifbrussel
  \newcommand{\siteurl}{brussel}
\fi
\ifgent
  \newcommand{\siteurl}{gent}
\fi

\macro{\cloudinfo}{\href{mailto:cloud@vscentrum.be}{cloud@vscentrum.be}}

\newcommand{\HPCManualURL}{https://hpcugent.github.io/vsc_user_docs/pdf/intro-HPC-\osurl-\siteurl.pdf}
\newcommand{\LinuxManualURL}{https://hpcugent.github.io/vsc_user_docs/pdf/intro-Linux-\osurl-\siteurl.pdf}
\newcommand{\VSCCloudManualURL}{https://hpcugent.github.io/vsc_user_docs/pdf/intro-Cloud-\osurl-\siteurl.pdf}

% Import the Site specific macros
\inputsite{macros}
% General macro for extracting the jobnumber from the jobid
\macro{\jobnumber}{\BeforeSubString{.}{\jobid}}
\macro{\pbsserver}{\BehindSubString{.}{\jobid}}

% Macro for 255 vs 256 diff in ssh-ed25519 fingerprint
\newcommand{\sshedfingerprintnote}{
\emph{Note:} it is possible that the \lstinline|ssh-ed25519| fingerprint starts with "\lstinline|ssh-ed25519 255|"
rather than "\lstinline|ssh-ed25519 256|" (or vice versa), depending on the PuTTY version you are using.
It is safe to ignore this \lstinline|255| versus \lstinline|256| difference, but the part after should be
\strong{identical}.
}

% Macro for winscp/putty fingerprint alert
\newcommand{\firsttimeconnection}{
The first time you make a connection to the login node, a Security
Alert will appear and you will be asked to verify the authenticity of the
login node.

Make sure the fingerprint in the alert matches one of the following:
\puttyFirstConnect

If it does, press \strong{\emph{Yes}}, if it doesn't, please contact \hpcinfo.

\sshedfingerprintnote

\begin{center}
\includegraphics*[width=3.42in]{ch2-putty-verify-authenticity}
\end{center}
}

\newcommand{\modulelocation}{/apps/\sitename/\defaultcluster/modules/all}
